\documentclass[11pt]{amsart}
\usepackage{amssymb,amsfonts}
\usepackage{euscript,mathrsfs}
\usepackage{latexsym}
\usepackage{xspace}
\usepackage{amscd}
\usepackage{amsmath,bbm}
\usepackage{color}
\usepackage{graphicx}
\usepackage[utf8]{inputenc}
\usepackage{amsmath,amsthm,amsopn,fixltx2e,mparhack,graphicx,microtype}
\usepackage{url}

\theoremstyle{plain}% default
\newtheorem{thm}{Theorem}
\newtheorem{lem}[thm]{Lemma}
\newtheorem{prop}[thm]{Proposition}
\newtheorem*{cor}{Corollary}
\newtheorem{algorithm}{Algorithm}

\theoremstyle{definition}
\newtheorem{defn}[thm]{Definition}
\newtheorem{conj}[thm]{Conjecture}
\newtheorem{exmp}[thm]{Example}

\theoremstyle{remark}
\newtheorem*{rem}{Remark}
\newtheorem*{note}{Note}
\newtheorem{case}{Case}

\newcommand{\Mac}{Macaulay2\xspace}
\newcommand{\SOS}{\textsc{SOS}\xspace}

\newcommand{\QQ}{\mathbb{Q}}
\newcommand{\RR}{\mathbb{R}}

\begin{document}

\title[SOS.m2]{Sums of squares in Macaulay2}

\author{Diego Cifuentes}
\address{Massachusetts Institute of Technology \\ Boston, USA}
\email{diegcif@mit.edu}

\author{Thomas Kahle}
\address{Otto-von-Guericke University \\ Magdeburg, Germany}
\email{thomas.kahle@ovgu.de}

\begin{abstract}
  The package \SOS implements sums of squares (SOS) decompositions in
  Macaulay2.  It is based on methods to rationalize SOS decompositions
  due to Parrilo and Peyrl.
\end{abstract}

\maketitle

% Some topics
% 1) brief introduction to SOS, nonnegativity, polynomial optimization
% 2) computing SOS decompositions (Motzkin, Robinson, Lax-Lax)
% 2) computing SOS decompositions of ternary forms (Harris)
% 3) parametrized SOS problems
% 4) SOS optimization (lowerBound, lasserreHierarchy)

\section{Getting started}

Let $k = \QQ$ or $k = \RR$ denote the rational or real numbers and let
$R = k[x_{1},\dots,x_{n}]$ be the polynomial ring.  An element
$f\in R$ is \emph{nonnegative} if $f(x) \ge 0$ for all
$x \in \RR^{n}$.  The element $f$ is a \emph{sum of squares} or
\emph{SOS} if there exist $h_{1},\dots,h_{m} \in R$ such that
$f=\sum_{i=1}^{m}h_{i}^{2}$.  Clearly, a sum of squares is
nonnegative, but not every nonnegative polynomial is a sum of squares.
Classically, Hilbert characterized in which cases sums of squares
coincide with nonnegative polynomials: univariate polynomials,
quadratic polynomials and bivariate quartics.  For an introduction to
the area we
recommend~\cite{scheiderer2009positivity,blekherman2012semidefinite}.

The purpose of the \SOS package is to deal with sums of squares in
Macaulay2.  The package contains the data type \verb|SOSPoly| to store
a sum of squares decomposition and perform basic operations on them.
The most basic methods take a polynomial and try to write it as a sum
of squares.  Here is an example of how to use the function
\verb|solveSOS|:
\begin{verbatim}
i6 : R = QQ[x,y];
i7 : f = 2*x^4+5*y^4-2*x^2*y^2+2*x^3*y;
i8 : sosPoly solveSOS f
Executing CSDP
Input file: /var/folders/73/jfd7r2r15zs03ks2lnx6zfpm0000gn/T/M2-92720-0/12.dat-s
Output file: /var/folders/73/jfd7r2r15zs03ks2lnx6zfpm0000gn/T/M2-92720-0/13
Status: SDP solved, primal-dual feasible

o8 = coeffs:
         43  231773
     {5, --, ------}
         20  344000
     gens:
         83 2    2  20 2         2
     {- ---x  + y , --x  + x*y, x }
        200         43

o8 : SOSPoly
\end{verbatim}
In this case the package is configured to use the SDP solver CSDP
which is called by \verb|solveSOS|.  The result is numerical, but the
package manages to round it to a rational solution.  The return value
of \verb|solveSOS| is an object \verb|SDPresult| which contains more
information.  Then \verb|sosPoly| extracts the sum of squares.  In
this case there are three two squares.  The package computed:
\[
  f = 5(-\frac{83}{200} x^{2}+y^{2})^{2} + \frac{43}{20}
  (\frac{20}{43}x^{2} + xy)^{2} + \frac{231773}{344000} (x^{2})^{2}
\]
Clearly this decomposition is not optimal, but at least it is fully
rational and simplifications can be done in post-processing.

... trace objective ... rounding tolerance ...

\subsection{History of the package}
The new version of \SOS was developed starting with version 1.5 of the
\SOS package due to Parrilo and Peyrl~\cite{peyrl2008computing}.  The
idea of rational reconstruction of SOS decompositions goes back to
that work.

\subsection{A word on SDP Solvers}
The \SOS package relies on numerical SDP solvers for many of its
operations.

\subsection{A small library of nonnegative forms}
The package implements a small library of nonnegative forms that have
names.  It is accessed with the function \verb|nonnegativeForm| which
takes the name of the polynomial and a suitable target ring.  The
following arguments are possible: \verb|Motzkin|, \verb|Robinson|,
\verb|Schmuedgen|, \verb|Lax-Lax|, \verb|Choi-Lam|, \verb|Scheiderer|,
and \verb|Harris|.  The package documentation contains more
information about these polynomials.  For a concrete example, the
Scheiderer polynomial is nonnegative, a sum of squares with real
polynomials, but not a sum of squares of rational polynomials.
\begin{verbatim}
i3 : R = QQ[x,y,z]
i4 : nonnegativeForm ("Scheiderer", R)

      4      3    4     2          2      2 2      3      3    4
o4 = x  + x*y  + y  - 3x y*z - 4x*y z + 2x z  + x*z  + y*z  + z

\end{verbatim}


\section{SOS decomposition of ternary forms}

According to a result of Hilbert, a ternary form $f\in k[x,y,z]$ can
be decomposed as a quotient of sums of squares using a simple
iterative algorithm \textbf{Will we have space to describe the
  algorithm?}:
\[
  f = \frac{p_{1}\cdots p_{s}}{q_{1}\cdots q_{t}} \qquad \text {
    $p_{i},q_{i}$ SOS}.
\]
This algorithm is implemented in in the function \verb|sosdecTernary|.
Here we apply it to the Motzkin polynomial.
\begin{verbatim}

\end{verbatim}

The \emph{Harris polynomial} is a ternary form with a record number
zeros. 
\begin{verbatim}
i3 : R = QQ[x,y,z]
i4 : nonnegativeForm ("Harris", {x,y,z})

        10      8 2      6 4      4 6      2 8      10      8 2      6 2 2      4 4 2
o4 = 16x   - 36x y  + 20x y  + 20x y  - 36x y  + 16y   - 36x z  + 57x y z  - 38x y z  +
     ----------------------------------------------------------------------------------
        2 6 2      8 2      6 4      4 2 4      2 4 4      6 4      4 6      2 2 6
     57x y z  - 36y z  + 20x z  - 38x y z  - 38x y z  + 20y z  + 20x z  + 57x y z  +
     ----------------------------------------------------------------------------------
        4 6      2 8      2 8      10
     20y z  - 36x z  - 36y z  + 16z
\end{verbatim}

At some point, SOS decompositions might be helpful in computing the
real radical of an ideal.  Namely,

\section{Optimization}




\section*{Acknowledgment}
\label{sec:acknowledgement}
The authors would like to thank the organizers and participants of the
\Mac\ workshop in May 2018 in Leipzig during which the renewal of the
SOS package started.

\bibliographystyle{amsplain}
\bibliography{sos}

\end{document}  

%%% Local Variables:
%%% mode: latex
%%% TeX-master: t
%%% End:
